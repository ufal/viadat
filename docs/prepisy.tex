\documentclass[10pt,a4paper]{article}
\usepackage[utf8x]{inputenc}
\usepackage{ucs}
\usepackage{amsmath}
\usepackage{amsfonts}
\usepackage{amssymb}
\usepackage{graphicx}

\title{\textsc{Viadat}: Pravidla pro formátování přepisů\\v1.0}
\date{}
\begin{document}
	\maketitle
\vspace{-1cm}

\begin{enumerate}
	\item Přepisy je možné vkládat ve formátech \texttt{.doc}, \texttt{.docx} nebo \texttt{.odt}.
	
	\item Jméno souboru s přepisem musí být, až na příponu, stejné jako jméno audio souboru. (Například: \texttt{Rozhovor s Janem Sokolem.odt} a \texttt{Rozhovor s Janem Sokolem.mp3}).
	
	\item Přepis může být rozdělen do více souborů. Soubory pak musí být  vzestupně očíslovány a název souboru pak musí končit číslem souboru.
	(Například: \texttt{Rozhovor s Janem Sokolem 1.odt}, \texttt{Rozhovor s Janem Sokolem 1.mp3}, \texttt{Rozhovor s Janem Sokolem 2.odt}, \texttt{Rozhovor s Janem Sokolem 2.mp3}). Soubor s přepisem musí pokrývat právě a pouze to, co je zaznamenáno v korespondujících audio souborech. Jinými slovy, rozdělení na soubory musí být na stejných místech jak v přepisech tak ve zvukových záznamech.  
	
	\item Hlavička přepisu musí mít jiný styl než text samotného přepisu (Například: rámeček nebo zarovnání na střed).
	Text mimo hlavičku smí obsahovat pouze samotný přepis.
	
	\item Text může být členěn do odstavců. V prvním odstavci a při změně řečníka je vždy nutné uvést řečníka na začátku odstavce.
	Pokud v novém odstavci zůstává řečník stejný jako v předchozím odstavci, není nutné řečníka uvádět. Jméno řečníka musí být odděleno dvojtečkou od vlastního textu. Například:\\
	Jan Novák: Dobrý den, ...
	
	\item Jméno řečníka musí být používáno ve \emph{stejné podobě} po celý přepis. Například nelze použít na jednom místě celé jméno a na jiném monogram.  
	
	\item V případě neznámých řečníků pojmenovávejte řečníky: ``Oso\-ba A'', ``Oso\-ba B'', atp. Pokud není jasné, kterému z řečníků náleží daný kus přepisu, použijte ,,Osoba ?''. 
	
	\item V případě části, kterou již není možné rozpoznat, vložte do textu řetězec ,,[nesrozumitelné]'' (včetně závorek, bez uvozovek).
	Řetězec [nesrozumitelné] může být vložen ve věte nebo jako samostatný odstavec.
	Odstavec obsahující pouze značku [nesrozumitelné] musí mít stejně jako ostatní odstavce uvedeného řečníka, pokud je jiný než pro předchozí odstavec.
\end{enumerate}

\end{document}